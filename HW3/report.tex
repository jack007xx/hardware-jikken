\documentclass[a4paper,10pt]{jsarticle}

% 数式
\usepackage{amsmath,amsfonts}
\usepackage{bm}
% 画像
\usepackage[dvipdfmx]{graphicx}

\usepackage{listingsutf8,jlisting} %日本語のコメントアウトをする場合jlistingが必要
%ここからソースコードの表示に関する設定
\lstset{
  basicstyle={\ttfamily},
  identifierstyle={\small},
  commentstyle={\smallitshape},
  keywordstyle={\small\bfseries},
  ndkeywordstyle={\small},
  stringstyle={\small\ttfamily},
  frame={tb},
  breaklines=true,
  columns=[l]{fullflexible},
  numbers=left,
  xrightmargin=0zw,
  xleftmargin=3zw,
  numberstyle={\scriptsize},
  stepnumber=1,
  numbersep=1zw,
  lineskip=-0.5ex
}

\begin{document}

\title{ハードウェア第3回レポート}
\author{坪井正太郎(101830245)}
\date{\today}
\maketitle

\section{課題3-1}
\subsection{概要}
この実験では,プログラム1を改変し,動作を確認することでシュミレータでの実験の基礎を体得する。

\subsection{実験方法}
\begin{lstlisting}[caption={program3-1},label={program3-1}]
  console.log("Start");
  // onoff.Gpio ライブラリをロード
  var Gpio = require("onoff").Gpio;
  // ピン番号を指定して Gpio インスタンスを作成
  // in / out は⼊⼒に使うか出⼒に使うかの指定
  1
  var led = new Gpio(4, "out");
  var button = new Gpio(12, "in");
  // 下段 LED1: 4, 17, 27, 22
  // 上段 LED1: 5, 6, 13, 19
  // ボタン 1: 20, ボタン 2: 16, ボタン 3: 12
  // 3 回までループ
  for(i=0; i<3; i++){
  // LED を 1 秒周期 (点灯 0.5 秒, 消灯 0.5 秒) で 5 回明滅
  for(j=0; j<5; j++){
  sleep(500);
  led.writeSync(1);
  sleep(500);
  led.writeSync(0);
  }
  // 無限ループでボタン⼊⼒を待つ (0: off, 1: on)
  console.log("[Waiting button input]");
  while(button.readSync() == 0) {
  sleep(100);
  }
  console.log("Pushed!!");
  }
  console.log("Finished!");
\end{lstlisting}

もとのプログラムを改変してledの配置を変更,追加する。
\begin{lstlisting}[caption={program3-1d},label={program3-1d}]
  console.log("Start");
  var Gpio = require("onoff").Gpio;

  var green_led1 = new Gpio(22, "out");
  var green_led2 = new Gpio(19, "out");
  var button = new Gpio(16, "in");
  // 下段 LED1: 4, 17, 27, 22
  // 上段 LED1: 5, 6, 13, 19
  // ボタン 1: 20, ボタン 2: 16, ボタン 3: 12
  for (i = 0; i < 3; i++) {
    // LED を 1 秒周期 (点灯 0.5 秒, 消灯 0.5 秒) で 5 回明滅
    for (j = 0; j < 5; j++) {
      sleep(500);
      green_led1.writeSync(1);
      green_led2.writeSync(1);
      sleep(500);
      green_led1.writeSync(0);
      green_led2.writeSync(0);
    }
    console.log("[Waiting button input]");
    while (button.readSync() == 0) {
      sleep(100);
    }
    console.log("Pushed!!");
  }
  console.log("Finished!");
\end{lstlisting}

\subsection{実験結果}
\ref{program3-1}を実行した結果LEDが明滅し,ボタン入力が正しく受け取られた。

\ref{program3-1d}を実行した結果,明滅するLEDが代わり,入力ボタンの位置も変わった。

\section{課題3-2-1}
% チャタリング

\section{課題3-2-2}
% プルアップダウン

\section{課題3-3}
\subsection{概要}
複数のLEDを同時に操作して,スイッチによる制御も行う。

\subsection{実験方法}
LEDを操作するオブジェクトの配列を作成し,countの増減でLEDの明滅順を制御する。
スイッチ入力のタイミングを増やすために,スリープを分割している。

\begin{lstlisting}[caption={program3-3},label={program3-3}]
  console.log("Start");
  var Gpio = require("onoff").Gpio;

  var led1 = new Gpio(4, "out");
  var led2 = new Gpio(17, "out");
  var led3 = new Gpio(27, "out");
  var led4 = new Gpio(22, "out");
  var leds = [led1, led2, led3, led4]

  var sw1 = new Gpio(20, "in");
  var sw2 = new Gpio(16, "in");
  // 下段 LED1: 4, 17, 27, 22
  // 上段 LED1: 5, 6, 13, 19
  // ボタン 1: 20, ボタン 2: 16, ボタン 3: 12
  var count = 0;
  var upCount = 1;
  while (true) {
    leds[count].writeSync(1);

    for (i = 0; i < 5; i++) {
      // スイッチ振り分け
      if (sw1.readSync() != 0) {
        console.log("stopped!!");
        while (sw1.readSync() == 0) {
          sleep(100);
        }
        console.log("restart!!");

      } else if (sw2.readSync() != 0) {
        console.log("reversed!!");
        upCount = (upCount == 1) ? 3 : 1;
      }
      sleep(100);
    }

    leds[count].writeSync(0);

    // upCountの値によって昇降切り替わる
    count = (count + upCount) % 4;
    console.log(count);
}
\end{lstlisting}

\subsection{実験結果}
明滅状態が並び順に推移されるようになった。
SW1を押したときには,推移が止まり入力待ち状態になった。
SW2を押したときには,推移の方向が逆順になった。

\subsection{考察}
スイッチ入力を1つでまとめると,タイミングが取れなくなるためスリープを分割して複数回入力を待つようにした。
これによって若干入力感度が向上した。

\section{課題3-4}
\subsection{概要}

\subsection{実験方法}
\begin{lstlisting}[caption={program3-4},label={program3-4}]
  
\end{lstlisting}


\subsection{実験結果}

\subsection{考察}

\section{課題3-5}


\end{document}
