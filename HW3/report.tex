\documentclass[a4paper,10pt]{jsarticle}

% 数式
\usepackage{amsmath,amsfonts}
\usepackage{bm}
% 画像
\usepackage[dvipdfmx]{graphicx}

\usepackage{listingsutf8,jlisting} %日本語のコメントアウトをする場合jlistingが必要
%ここからソースコードの表示に関する設定
\lstset{
  basicstyle={\ttfamily},
  identifierstyle={\small},
  commentstyle={\smallitshape},
  keywordstyle={\small\bfseries},
  ndkeywordstyle={\small},
  stringstyle={\small\ttfamily},
  frame={tb},
  breaklines=true,
  columns=[l]{fullflexible},
  numbers=left,
  xrightmargin=0zw,
  xleftmargin=3zw,
  numberstyle={\scriptsize},
  stepnumber=1,
  numbersep=1zw,
  lineskip=-0.5ex
}

\begin{document}

\title{ハードウェア第3回レポート}
\author{坪井正太郎(101830245)}
\date{\today}
\maketitle

\section{課題3-1}
\subsection{概要}
この実験では,プログラム1を改変し,動作を確認することでシュミレータでの実験の基礎を体得する。

\subsection{実験方法}
\begin{lstlisting}[caption={program3-1},label={program3-1}]
  console.log("Start");
  // onoff.Gpio ライブラリをロード
  var Gpio = require("onoff").Gpio;
  // ピン番号を指定して Gpio インスタンスを作成
  // in / out は⼊⼒に使うか出⼒に使うかの指定
  1
  var led = new Gpio(4, "out");
  var button = new Gpio(12, "in");
  // 下段 LED1: 4, 17, 27, 22
  // 上段 LED1: 5, 6, 13, 19
  // ボタン 1: 20, ボタン 2: 16, ボタン 3: 12
  // 3 回までループ
  for(i=0; i<3; i++){
  // LED を 1 秒周期 (点灯 0.5 秒, 消灯 0.5 秒) で 5 回明滅
  for(j=0; j<5; j++){
  sleep(500);
  led.writeSync(1);
  sleep(500);
  led.writeSync(0);
  }
  // 無限ループでボタン⼊⼒を待つ (0: off, 1: on)
  console.log("[Waiting button input]");
  while(button.readSync() == 0) {
  sleep(100);
  }
  console.log("Pushed!!");
  }
  console.log("Finished!");
\end{lstlisting}

もとのプログラムを改変してledの配置を変更,追加する。
\begin{lstlisting}[caption={program3-1d},label={program3-1d}]
  console.log("Start");
  var Gpio = require("onoff").Gpio;

  var green_led1 = new Gpio(22, "out");
  var green_led2 = new Gpio(19, "out");
  var button = new Gpio(16, "in");
  // 下段 LED1: 4, 17, 27, 22
  // 上段 LED1: 5, 6, 13, 19
  // ボタン 1: 20, ボタン 2: 16, ボタン 3: 12
  for (i = 0; i < 3; i++) {
    // LED を 1 秒周期 (点灯 0.5 秒, 消灯 0.5 秒) で 5 回明滅
    for (j = 0; j < 5; j++) {
      sleep(500);
      green_led1.writeSync(1);
      green_led2.writeSync(1);
      sleep(500);
      green_led1.writeSync(0);
      green_led2.writeSync(0);
    }
    console.log("[Waiting button input]");
    while (button.readSync() == 0) {
      sleep(100);
    }
    console.log("Pushed!!");
  }
  console.log("Finished!");
\end{lstlisting}

\subsection{実験結果}
\ref{program3-1}を実行した結果LEDが明滅し,ボタン入力が正しく受け取られた。

\ref{program3-1d}を実行した結果,明滅するLEDが代わり,入力ボタンの位置も変わった。

\section{課題3-2-1}
\subsection{チャタリングの発生要因}
物理的な接点同士の接触時に,接点が激突してバウンドすることをチャタリングという。
これによって,バウンドが収まるまでスイッチがON,OFFを繰り返すことになる。
このバウンドはおよそ1〜10mSecほどの時間がかかることが多い。

\subsection{チャタリングの軽減}
クロック信号を出力したい場合など,ソフト的な処理ができない場合,CRのフィルタによって電圧上昇を遅らせる。
通常のアースに加えて,コンデンサを接続したアースを抵抗と出力先の間につける。
これによってコンデンサに充電されるまでは緩やかに電圧が上昇し,クロックの立ち下がりのときにはコンデンサの放電が行われるので降下も緩やかになる。
物理スイッチのチャタリングが収まる時間より大きな時定数をもつコンデンサと抵抗の組み合わせを選ぶことで,チャタリングを軽減させることができる。



\section{課題3-2-2}
% プルアップダウン

\section{課題3-3}
\subsection{概要}
複数のLEDを同時に操作して,スイッチによる制御も行う。

\subsection{実験方法}
LEDを操作するオブジェクトの配列を作成し,countの増減でLEDの明滅順を制御する。
スイッチ入力のタイミングを増やすために,スリープを分割している。

\begin{lstlisting}[caption={program3-3},label={program3-3}]
  console.log("Start");
  var Gpio = require("onoff").Gpio;

  var led1 = new Gpio(4, "out");
  var led2 = new Gpio(17, "out");
  var led3 = new Gpio(27, "out");
  var led4 = new Gpio(22, "out");
  var leds = [led1, led2, led3, led4]

  var sw1 = new Gpio(20, "in");
  var sw2 = new Gpio(16, "in");
  // 下段 LED1: 4, 17, 27, 22
  // 上段 LED1: 5, 6, 13, 19
  // ボタン 1: 20, ボタン 2: 16, ボタン 3: 12
  var count = 0;
  var upCount = 1;
  while (true) {
    leds[count].writeSync(1);

    for (i = 0; i < 5; i++) {
      // スイッチ振り分け
      if (sw1.readSync() != 0) {
        console.log("stopped!!");
        while (sw1.readSync() == 0) {
          sleep(100);
        }
        console.log("restart!!");

      } else if (sw2.readSync() != 0) {
        console.log("reversed!!");
        upCount = (upCount == 1) ? 3 : 1;
      }
      sleep(100);
    }

    leds[count].writeSync(0);

    // upCountの値によって昇降切り替わる
    count = (count + upCount) % 4;
    console.log(count);
}
\end{lstlisting}

\subsection{実験結果}
明滅状態が並び順に推移されるようになった。
SW1を押したときには,推移が止まり入力待ち状態になった。
SW2を押したときには,推移の方向が逆順になった。

\subsection{考察}
スイッチ入力を1つでまとめると,タイミングが取れなくなるためスリープを分割して複数回入力を待つようにした。
これによって若干入力感度が向上した。

\section{課題3-4}
\subsection{概要}
この実験では,4つのLEDから構成されるステッピングモータを制御する。
その際,励磁方法と回転方向も制御できるようなコードを実行し,動作を確認する。

\subsection{実験方法}
コメントアウト部のテーブルを差し替えて,励磁方法を変更できる。

\begin{lstlisting}[caption={program3-4},label={program3-4}]
  console.log("Start");
  var Gpio = require("onoff").Gpio;

  var coil1 = new Gpio(4, "out");
  var coil2 = new Gpio(17, "out");
  var coil3 = new Gpio(27, "out");
  var coil4 = new Gpio(22, "out");
  var coils = [coil1, coil2, coil3, coil4]

  var sw1 = new Gpio(20, "in");
  var sw2 = new Gpio(16, "in");
  // 下段 coil1: 4, 17, 27, 22
  // 上段 coil1: 5, 6, 13, 19
  // ボタン 1: 20, ボタン 2: 16, ボタン 3: 12

  var count = 0;
  var isClockwise = true;

  // 1相励磁
  var excitationTable = [[1, 0, 0, 0], [0, 1, 0, 0], [0, 0, 1, 0], [0, 0, 0, 1]];
  // 2相励磁
  // var excitationTable = [[1, 1, 0, 0], [0, 1, 1, 0], [0, 0, 1, 1], [1, 0, 0, 1]];
  // 1-2相励磁
  // var excitationTable = [[1, 0, 0, 0], [1, 1, 0, 0], [0, 1, 0, 0], [0, 1, 1, 0], [0, 0, 1, 0], [0, 0, 1, 1], [0, 0, 0, 1], [1, 0, 0, 1]];

  while (true) {
    // ここから
    for (i = 0; i < coils.length; i++) {
      coils[i].writeSync(excitationTable[count][i]);
    }
    // ここまで同時

    for (i = 0; i < 5; i++) {
      // スイッチ振り分け
      if (sw1.readSync() != 0) {
        console.log("stopped!!");
        while (sw1.readSync() == 0) {
          sleep(100);
        }
        console.log("restart!!");

      } else if (sw2.readSync() != 0) {
        console.log("reversed!!");
        isClockwise = !isClockwise;
      }
      sleep(100);
    }

    // isClockwiseによって昇降切り替わる
    count = (count + ((isClockwise) ? 1 : excitationTable.length - 1)) % excitationTable.length;
    console.log(count);
}

\end{lstlisting}

\subsection{実験結果}
各励磁方法に対応した明滅状態で推移した。
SW1で停止,開始を,SW2で正転,逆転を切り替えることができた。

\section{課題3-5}
\subsection{概要}
この実験では,2つのステッピングモータを制御する。

SW1では回転と停止の制御を,SW2,SW3でそれぞれのステッピングモータの正逆転を切り替える。
これはロボットカーの制御を模している。

\subsection{実験方法}
コイルの配列を2つ用意する他は3-4と同じ。
それぞれに対して正逆転のフラグを用意して,SW2,3で切り替える。

\begin{lstlisting}[caption={program3-5},label={program3-5}]
  console.log("Start");
  var Gpio = require("onoff").Gpio;

  var coil1 = new Gpio(4, "out");
  var coil2 = new Gpio(17, "out");
  var coil3 = new Gpio(27, "out");
  var coil4 = new Gpio(22, "out");
  var coils1 = [coil1, coil2, coil3, coil4]

  var coil5 = new Gpio(5, "out");
  var coil6 = new Gpio(6, "out");
  var coil7 = new Gpio(13, "out");
  var coil8 = new Gpio(19, "out");
  var coils2 = [coil5, coil6, coil7, coil8]

  var sw1 = new Gpio(20, "in");
  var sw2 = new Gpio(16, "in");
  var sw3 = new Gpio(12, "in");
  // 下段 coil1: 4, 17, 27, 22
  // 上段 coil1: 5, 6, 13, 19
  // ボタン 1: 20, ボタン 2: 16, ボタン 3: 12

  var count1 = 0;
  var count2 = 0;
  var isClockwise1 = true;
  var isClockwise2 = true;

  // 1相励磁
  var excitationTable = [[1, 0, 0, 0], [0, 1, 0, 0], [0, 0, 1, 0], [0, 0, 0, 1]];
  // 2相励磁
  // var excitationTable = [[1, 1, 0, 0], [0, 1, 1, 0], [0, 0, 1, 1], [1, 0, 0, 1]];
  // 1-2相励磁
  // var excitationTable = [[1, 0, 0, 0], [1, 1, 0, 0], [0, 1, 0, 0], [0, 1, 1, 0], [0, 0, 1, 0], [0, 0, 1, 1], [0, 0, 0, 1], [1, 0, 0, 1]];

  console.log("[Waiting button input]");
  while (sw1.readSync() == 0) {
    sleep(100);
  }
  console.log("start!!");

  while (true) {
    // ここから
    for (i = 0; i < coils1.length; i++) {
      coils1[i].writeSync(excitationTable[count1][i]);
    }
    for (i = 0; i < coils2.length; i++) {
      coils2[i].writeSync(excitationTable[count2][i]);
    }
    // ここまで同時

    for (i = 0; i < 5; i++) {
      // スイッチ振り分け
      if (sw1.readSync() != 0) {
        console.log("stopped!!");
        isClockwise1 = true;
        isClockwise2 = true;
        while (sw1.readSync() == 0) {
          sleep(100);
        }
        console.log("restart!!");

      } else if (sw2.readSync() != 0) {
        console.log("turn right");
        isClockwise1 = !isClockwise1;

      } else if (sw3.readSync() != 0) {
        console.log("turn left");
        isClockwise2 = !isClockwise2;

      }
      sleep(100);
    }

    // isClockwiseによって昇降切り替わる
    count1 = (count1 + ((isClockwise1) ? 1 : excitationTable.length - 1)) % excitationTable.length;
    count2 = (count2 + ((isClockwise2) ? 1 : excitationTable.length - 1)) % excitationTable.length;
  }
\end{lstlisting}

\subsection{実験結果}
SW1を押すと開始と停止を切り替えることができた。
SW2とSW3によって各LEDの推移順を独立に操作できた。

\end{document}
